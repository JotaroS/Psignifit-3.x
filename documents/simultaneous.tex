\documentclass[a4paper,11pt]{scrartcl}

\usepackage{amsmath,amssymb}
\usepackage[utf8]{inputenc}

\title{Informed priors and quasi-simultaneous inference for psychometric functions}
\author{Ingo Fründ}
\date{\today}

\def\prob{\mathbb{P}}

\begin{document}

\maketitle
% Your text here
When fitting psychometric functions, we are often interested in summarizing the performance of observers in as few numbers as possible.
Of particular interest is a summary of the performance that consists of a single parameter, the threshold.
Unfortunately, comparisons of psychophysical thresholds are only meaningful if other parameters of the psychometric functions --- such as slope and lapse rate --- are constant across conditions.
Here, I demonstrate how priors can be derived that incorporate the information from other conditions.
If these priors are used in bayesian inference of psychometric functions, this allows for quasi-simultaneous fits that constrain parameters to be constant across conditions.

\section{A simple example}

We start with a simple example to illustrate the basic idea behind the approach.
Assume, we colleted data $X_i$ under two different conditions $i\in\{1,2\}$.
We want to model these data depending on a parameter $\theta$.
In principle, $\theta$ might differ between the two conditions, but we will assume here, that $\theta_1=\theta_2$.
In this particular case, we can write down the posterior distribution of $\theta$:
%
$$
\prob(\theta|X_1,X_2) \propto \prob(X_1,X_2|\theta) \prob(\theta) = \prob(X_1|\theta)\prob(X_2|\theta)\prob(\theta),
$$
%
Where the second equality holds under the assumption that the two conditions are independent.

In practical applications, we will typically not be able to write down $\prob(\theta|X_1,X_2)$ directly.
Thus, we will have to resort to sampling strategies to explore $\prob(\theta|X_1,X_2)$.
The strategy in this case is therefore to perform two sampling stages.
In the first sampling run, we derive proper priors for the parameter $\theta$ and in the second sampling run, we use these priors to perform the true analysis.

For the first sampling run, we employ an improper uniform prior, that is $\prob(\theta)=\kappa$ for all $\theta$ and a fixed $\kappa$.
We then use MCMC to generate a sample $(\theta_{1i})_{i=1}^m$ of size $m$ from $\prob(\theta|X_1)\propto\prob(X_1|\theta)\cdot\kappa$ and another sample
$(\theta_{2i})_{i=1}^m$ from $\prob(\theta|X_2)\propto\prob(X_2|\theta)\cdot\kappa$.
We now fit a parametric model to each sample and use these model as priors for the second sampling run.
Denote the parametric model fitted to $(\theta_{1i})$ by $f_1$ and the parametric model fitted to $(\theta_{2i})$ by $f_2$.
Then approximate samples from the simultaneous model can be obtained by sampling either
%
$$
\prob(\theta|X_1,X_2) \propto\prob(X_1|\theta)f_2(\theta),
$$
%
or
%
$$
\prob(\theta|X_1,X_2) \propto\prob(X_2|\theta)f_1(\theta).
$$
%

\section{Constant parameters}

I will now expand the above strategy to the case in which the model has more than a single parameter.
Let us think about a model with three parameters, $\theta$, $\vartheta_1$, $\vartheta_2$.
As the notation already suggests,  $\theta$ shall be assumed fixed across conditions, while $\vartheta_1$ and $\vartheta_2$ depend on the
experimental condition.
Again, we write down the posterior distribution
%
\begin{align*}
\prob(\theta,\vartheta_1,\vartheta_2|X_1,X_2) &=\prob(\theta|X_1,X_2)\prob(\vartheta_1,\vartheta_2|X_1,X_2)\\
&= \prob(\theta|X_1,X_2)\prob(\vartheta_1|X_1)\prob(\vartheta_2|X_2),
\end{align*}
%
Where the first equality requires $\theta$ to be independent of $\vartheta_1$ and $\vartheta_2$.
That means, the parameter we want constraint to be equal across conditions necessarily has to be independent of the other parameters!
The second equality is valid if $\vartheta_1$ depends only on the data $X_1$ and $\vartheta_2$ depends only on the data $X_2$.
This is a rather plausible condition.

We can now use the same trick as above to derive priors for $\theta$ by drawing samples $(\theta_{1i})_{i=1}^m$ from $\prob(\theta,\vartheta_1|X_1)\cdot\kappa$
and $(\theta_{2i})_{i=1}^m$ from $\prob(\theta,\vartheta_2|X_2)\cdot\kappa$.
Then, we fit the marginal distributions of $(\theta_{1i})$ and $(\theta_{2i})$ to obtain expressions for $f_1$ and $f_2$.
In a second sampling run, we can then perform quasi simultaneous sampling using
%
$$
\prob(\theta,\vartheta_1|X_1,X_2) \propto\prob(X_1|\theta,\vartheta_1)f_2(\theta)\prob(\vartheta_1),
$$
%
and
%
$$
\prob(\theta,\vartheta_2|X_1,X_2) \propto\prob(X_2|\theta,\vartheta_2)f_1(\theta)\prob(\vartheta_2).
$$
%

\section{More than two conditions}

The extension of these ideas to cases with more than two conditions is straight forward:
In the first sampling run, we determine the prior densities $f_j, j=1,\dots,n$ from each dataset $X_j$ in isolation.
We then combine the $f_j$ in order to sample
%
\begin{equation}
    \label{eq:allcombined}
\prob(\theta,\vartheta_j|X_j, j=1,\dots,n) \propto\prob(X_j|\vartheta_j,\theta) \prob(\vartheta_j) \prod_{\underset{\ell\neq j}{\ell=1}}^n f_\ell ( \theta ),
\end{equation}
%
in isolation for each $j=1,\dots,n$.
Thus, the product $\prod_{\underset{\ell\neq j}{\ell=1}}^n f_\ell ( \theta )$ can be taken as a prior that incorporates prior knowledge about the other psychometric functions in the design.

\section{Specific choices of fitted posteriors}

Here, we will highlight reasonable choices to fit the posterior distributions and we will demonstrate, how these choices imply a closed functional form for the product prior in equation \eqref{eq:allcombined}.

\subsection{Fitting posterior distributions for scale parameters}

In order to compare, thresholds obtained from different conditions, it is typically required that the slopes across all these conditions are equal.
Although threshold comparisons are possible with unequal slopes, too, it is far more complicated to arrive at valid conclusions from these comparisons.
In psignifit, many parameterizations of the psychometric functions do not explicitely specify the slope of the psychometric function but instead have a parameter that is somehow related to the inverse slope.
Posterior distributions for this parameter often resemble a Gamma distribution, i.e. the seem to have a density of the form
%
\begin{equation}
    \label{eq:GammaDensity}
    f ( x | k, \theta ) = x^{k-1} \frac{\exp(-x/\theta)}{\Gamma(k)\theta^k}.
\end{equation}
%
We now assume that we have fitted this density (using for instance maximum likelihood) to a set of $n$ sampled posteriors from different conditions and we want to derive the product prior for the $n+1$-th condition.
The product prior has the density
%
\begin{align*}
    \bar{f}(x) &= \prod_{j=1}^n f_j(x) = \prod_{j=1}^n x^{k_j-1} \frac{\exp(-x/\theta_j)}{\Gamma(k_j)\theta_j^{k_j}} \\
    &\propto \prod_{j=1}^n x^{k_j-1} \exp(-x/\theta_j) \\
    &= x^{\sum_{j=1}^n k_j-1} \exp(-x\sum_{j=1}^n \frac{1}{\theta_j}) \\
    &= x^{\bar{k}-1} \exp(-x/\bar{\theta}),
\end{align*}
%
where we set
%
$$
\bar{k} := 1 + \sum_{j=1}^n k_j-1,
$$
%
and
%
$$
\bar{\theta} := \Big( \sum_{j=1}^n \frac{1}{\theta_j} \Big)^{-1}.
$$
%
Thus, the product prior is again a Gamma distribution with parameters $\bar{k}$ and $\bar{\theta}$!

\subsection{Fitting posterior distributions for rates}

The psychometric function model includes two rate parameters, the guessing rate $\gamma$ that describes the lower asymptote and the lapse rate $\lambda$ that describes the upper asymptote.
A very reasonable model for rates is the Beta distribution (cite Ferrari?).
The Beta distribution lives on the unit interval $(0,1)$ and takes the density
%
\begin{equation}
    f ( x | \alpha, \beta ) = \frac{\Gamma(\alpha+\beta)}{\Gamma(\alpha)\Gamma(\beta)} x^{\alpha-1} (1-x)^{\beta-1}.
    \label{eq:BetaDensity}
\end{equation}
%
Here, we can use the same trick as with the Gamma distribution to derive a sensible product prior:
We assume that we fitted $n$ sets of parameters $(\alpha_j,\beta_j)_{j=1}^n$ and want to combine these to obtain a prior for the $n+1$-th condition.
The product prior in this case is
%
\begin{align*}
    \bar{f}(x) &= \prod_{j=1}^n f_j(x) = \prod_{j=1}^n \frac{\Gamma(\alpha_j+\beta_j)}{\Gamma(\alpha_j)\Gamma(\beta_j)} x^{\alpha_j-1} (1-x)^{\beta_j-1}\\
    &\propto \prod_{j=1}^n x^{\alpha_j-1} (1-x)^{\beta_j-1} \\
    &= x^{\sum_{j=1}^n \alpha_j-1} (1-x)^{\sum_{j=1}^n\beta_j-1} \\
    &= x^{\bar{\alpha}-1} (1-x)^{\bar{\beta}-1},
\end{align*}
%
where $\bar{\alpha}:= 1+ \sum_{j=1}^n\alpha_j-1$ and $\bar{\beta}:=1+\sum_{j=1}^n\beta_j-1$.

Thus, the product prior in this case, is again a Beta distribution, with parameters $\bar{\alpha}$ and $\bar{\beta}$.

\subsection{Fitting posterior distributions for thresholds}

In some, very rare situations, we might also be interested in having equal posterior distributions for thresholds.
A reasonable approximation for posterior threshold distributions is often the Normal distribution.
Interestingly, the procedure used to derive the product prior in the preceeding paragraphs does not work well for the Normal distribution.
Assume, we fitted the posteriors in conditions $j=1,\dots,n$ with Normal densities $(f_j)_{j=1}^n$ with parameters $(\mu_j,\sigma_j)_{j=1}^n$.
The product is then
%
\begin{align*}
    \bar{f}(x) &= \prod_{j=1}^n f_j(x) = \prod_{j=1}^n \frac{1}{\sqrt{2\pi}\sigma_j} \exp ( - \frac{(x-\mu_j)^2}{2\sigma_j^2})\\
    &\propto \prod_{j=1}^n \exp ( - \frac{(x-\mu_j)^2}{2\sigma_j^2})\\
    &= \exp ( - \sum_{j=1}^n \frac{(x-\mu_j)^2}{2\sigma_j^2} ) \\
    &= \exp ( - \frac{1}{2} \big( \sum_{j=1}^n \frac{x^2}{\sigma_j^2} - \sum_{j=1}^n 2\frac{x\mu_j}{\sigma_j^2} + \sum_{j=1}^n \frac{\mu_j^2}{\sigma_j^2} \big) )\\
    &\propto\exp ( - \frac{1}{2} \big( \sum_{j=1}^n \frac{x^2}{\sigma_j^2} - \sum_{j=1}^n 2\frac{x\mu_j}{\sigma_j^2}\big) ) \\
    &= \exp ( - \frac{1}{2} \big( x^2\sum_{j=1}^n \frac{1}{\sigma_j^2} - 2x\sum_{j=1}^n 2\frac{\mu_j}{\sigma_j^2} \big) ) \\
    &= \exp ( - \frac{1}{2} \underbrace{\sum_{j=1}^n \frac{1}{\sigma_j^2}}_{=:\bar{\sigma}^{-2}} \big( x^2 - 2x\underbrace{\frac{\sum_{j=1}^n \frac{\mu_j}{\sigma_j^2}}{\sum_{j=1}^n \frac{1}{\sigma_j^2}}}_{=:\bar{\mu}}\big) ) \\
    &\propto \frac{1}{\sqrt{2\pi}\bar{\sigma}} \exp( - \frac{(x-\bar{\mu})^2}{2\bar{\sigma}^2}).
\end{align*}
%
Thus, comparing coefficients with the Normal density, we end up with a new Normal distribution with parameters
%
$$
\bar{\sigma}^2 := \big(\sum_{j=1}^n \frac{1}{\sigma_j^2}\big)^{-1},
$$
%
$$
\bar{\mu} := \frac{\sum_{j=1}^n\frac{\mu_j}{\sigma_j^2}}{\sum_{j=1}^n\frac{1}{\sigma_j^2}}.
$$
%
\end{document}
