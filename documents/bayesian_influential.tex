\documentclass[a4paper,11pt]{scrartcl}

\usepackage{amsmath,amssymb}
\usepackage[utf8]{inputenc}

\title{Determining influential observations using markov chain monte carlo samples}
\author{Ingo Fründ}
\date{November 9th, 2009}

\def\E{\mathbb{E}}

\begin{document}

\maketitle
% Your text here
Assume we have obtained response counts $\mathbf{y} = (y_k)_{k=1}^K$ in $K$ blocks.
We model these response counts using a parameter vector $\theta$.
We write the posterior density of $\theta$ given the data as
%
$$
p(\theta|\mathbf{y}) =: f(\theta).
$$
%
We are now interested in determining wether a particular block $k_0$ is an influential
observation.
To this end, we want to determine the Kullbach-Leibler divergence between the posterior
density of $\theta$ given all blocks $p(\theta|\mathbf{y})$ and the posterior density of
$\theta$ given all blocks except for block $k_0$.
We will write the posterior density of $\theta$ given all blocks except for block $k_0$
as
%
$$
p(\theta|\mathbf{y}_{-k_0}) =: g(\theta).
$$
%
We don't have direct access to $f$ and $g$.
We can write
%
$$
f(\theta) = \frac{1}{Z_f} \tilde{f} = \frac{1}{Z_f} \exp(-F(\theta)),\quad g(\theta) = \frac{1}{Z_g} \tilde{g} = \frac{1}{Z_g} \exp(-G(\theta)).
$$
%
The psignifit C++ engine allows fast evaluation of $-F(\theta)$ and $-G(\theta)$ via the
\verb!neglpost()! method of the \verb!PsiPsychometric! class.
In chapter 11 of the book by Bishop we learn that for samples $\theta^{(\ell)}, \ell=1,\dots,N$ from
the density $f$, we can obtain the ratio of the partition functions using importance sampling
%
$$
\frac{Z_g}{Z_f} \approx \frac{1}{N} \sum_{\ell=1}^N \exp(-F(\theta^{(\ell)} + G(\theta^{(\ell)}) ),
$$
%
where in our case, we can use the MCMC samples from the full data set fit.
Importance sampling is efficient if $F$ and $G$ are reasonably similar.
We assume that this is the case.
The less similar $F$ and $G$, the higher will be the Kullbach-Leibler divergence and the more likely
will it be that block $k_0$ is an influential observation anyhow.

The Kullbach-Leibler divergence of $f$ and $g$ is
%
\begin{align*}
    D_{KL}(f|g) := \E_f (\log \frac{f}{g} ) &= \E_f ( \log \frac{\tilde{f}}{Z_f}\cdot \frac{Z_g}{\tilde{g}} )
    = \E_f ( \log \frac{\tilde{f}}{\tilde{g}} + \log \frac{Z_g}{Z_f} )
    =\E_f ( \log \frac{\tilde{f}}{\tilde{g}} ) + \log \frac{Z_g}{Z_f}\\
    &\approx \frac{1}{N} \sum_{\ell=1}^N \log \frac{\tilde{f}(\theta^{(\ell)})}{\tilde{g}(\theta^{(\ell)})} +
    \log \Big( \frac{1}{N} \sum_{\ell=1}^N \exp(-F(\theta^{(\ell)})) + G(\theta^{(\ell)}) \Big) \\
    &= \frac{1}{N} \sum_{\ell=1}^N \log \frac{\exp(-F(\theta^{(\ell)}))}{\exp(-G(\theta^{(\ell)}))} +
    \log \Big( \frac{1}{N} \sum_{\ell=1}^N \exp(-F(\theta^{(\ell)})) + G(\theta^{(\ell)}) \Big) \\
    &= \frac{1}{N} \Big(\sum_{\ell=1}^N -F(\theta^{(\ell)})) + G(\theta^{(\ell)})\Big) +
    \log \Big( \frac{1}{N} \sum_{\ell=1}^N \exp(-F(\theta^{(\ell)})) + G(\theta^{(\ell)}) \Big).
\end{align*}
%
This can be evaluated without sampling from $G$ at all.
We just need to store $-F(\theta^{(\ell)}) + G(\theta^{(\ell)})$ with each sample $\theta^{(\ell)}$.

\end{document}
