\documentclass[a4paper,11pt]{scrartcl}

\usepackage{amsmath,amssymb}
\usepackage[utf8]{inputenc}

\title{Correction of confidence limits for bad observers}
\author{Ingo Fründ}
\date{\today}

\def\Binomial{\operatorname{Binom}}
\def\Beta{\operatorname{Beta}}
\def\p{\partial}

\begin{document}

\maketitle
% Your text here
Assume we have sampled detection data from an observer at $m$ different intensity levels $(x_i)_{i=1}^m$.
At each intensity level a number of $n_i$ trials was presented and the observer made $k_i$ correct responses (or reported $k_i$ detections or \dots), $i=1,\dots,m$.
Typically, we would model $k_i$ using a binomial distribution as
%
\begin{equation}
    \label{eq:1}
    k_i \sim \Binomial ( n_i, \Psi( x_i | \theta ) ).
\end{equation}
%
This is a useful model to obtain point estimates of the parameter vector $\theta$.
In using equation \eqref{eq:1} we implicitely assume that the $n_i$ trials at stimulus level $x_i$ are statistically independent.
Unfortunately, this is not necessarily the case in real psychophysical data.

If trials depend on each others, the effective number of trials will be lower:
With sufficiently strong dependencies, it will eventually be possible to predict the outcome on some trials from the outcome on other trials.
In that case, the predicted trials will add no further information to the estimated parameter vector $\theta$.
In particular estimates of the variance of $\theta$ are based on the number of trials that added information to the estimation process.
If there have been less trials than we actually thought, the variance will be underestimated.
I am going to present a method to estimate the reduction of independent trials and suggest two ways to correct credibility intervals for $\theta$.

We believe that instead of $n_i$ trials, we actually only recorded $\nu n_i$ trials at stimulus level $x_i$, $\nu\in(0,1)$.
We assume that inter-trial dependencies do not depend on the stimulus level $x_i$.
Thus, we assume a scalar factor $\nu$.

To determine $\nu$, we consider residuals of the model \eqref{eq:1}:
%
$$
r_i = \frac{k_i}{n_i} - \Psi ( x_i | \theta )
$$
%
A sensible model for the residuals is the beta distribution,
%
$$
\Beta ( \alpha_i, \beta_i ),\quad\alpha_i,\beta_i>0.
$$
%
The beta distribution $\Beta (\alpha,\beta)$, can be interpreted as the posterior distribution for the success probability of a binomial model
after observing $\alpha-1$ successes and $\beta-1$ failures.
We can thus write from equation \eqref{eq:1} (with the number of trials as $\nu n_i$)
%
\begin{equation}
    p_i := \frac{k_i}{n_i} \sim \Beta ( \alpha_i, \beta_i ) = \Beta \big( \Psi(x_i|\theta)\nu n_i, (1-\Psi(x_i|\theta))\nu n_i\big).
    \label{eq:2}
\end{equation}
%
If we know a value for $\theta$, this distribution still depends on one free parameter $\nu$.
To derive a correction factor $\nu$, we can fix $\theta=\hat{\theta}$ at the (constrained) maximum likelihood estimate of $\theta$ based
on the assumption of binomial variance, i.e. equation \eqref{eq:1}.
The log likelihood function for $\nu$ is then
%
\begin{equation}
    \begin{split}
\ell (\nu) &= \sum_{i=1}^m \log ( f ( p_i; \hat{\Psi}_i, \nu n_i) ) \\
&= \sum_{i=1}^m \log ( \Gamma ( \nu n_i+2)) - \log ( \Gamma ( \hat{\Psi}_i \nu n_i +1 ) ) - \log ( \Gamma ( (1-\hat{\Psi}_i) \nu n_i + 1 )) \\
&\qquad + \hat{\Psi}_i \nu n_i \log (p_i) + (1-\hat{\Psi}_i) \nu n_i \log ( 1-p_i).
\end{split}
\label{eq:3}
\end{equation}
%
Here, we set $f: (0,1) \to \mathbb{R}^+$ as the densitiy of the beta distribution parameterized in terms of observed success rate and total number of trials,
and $\hat{\Psi}_i := \Psi ( x_i | \hat{\theta} )$.

Analytical maximization of $\ell ( \nu)$ is difficult due to the $\Gamma$ function.
In particular, the derivative of equation \eqref{eq:3} involves the digamma function of 0-th order $\psi_0 = \frac{d}{dx} \log\Gamma(x)$:
%
\begin{equation}
    \begin{split}
        \frac{\p\ell}{\p \nu} (\nu) &= \sum_{i=1}^m n\psi_0 ( \nu n_i + 2 )
            - \hat{\Psi}_i n_i \psi_0 ( \nu \hat{\Psi}_i n_i + 1)
            - (1-\hat{\Psi}_i) n_i \psi_0 ( \nu (1-\hat{\Psi}_i) n_i + 1) \\
            &\qquad + \hat{\Psi}_i n_i \log (p_i) + (1-\hat{\Psi}_i) n_i log (1-p_i).
    \end{split}
    \label{eq:4}
\end{equation}
%
To find a maximum of $\ell$, we can use newton iterations of the form
%
$$
\nu_{k+1} = \nu_{k} + \frac{\ell\prime(\nu_k)}{\ell\prime\prime(\nu_k)},
$$
%
where $\ell\prime := \p\ell/\p\nu$ and
%
\begin{equation}
    \begin{split}
        \ell\prime\prime ( \nu ) &:= \frac{\p^2\ell}{\p\ell^2} ( \nu) \\
        &= \sum_{i=1}^m n_i^2 \psi_1 ( \nu n_i + 2 ) - \hat{\Psi}_i^2 n_i^2 \psi_1 ( \nu \hat{\Psi}_i n_i + 1 ) - (1-\hat{\Psi}_i)^2 n_i^2 \psi_1 ( \nu (1-\hat{\Psi}_i) n_i + 1 )
    \end{split}
    \label{eq:5}
\end{equation}

We can use the estimated value $\nu^*$ in two ways to correct the confidence intervals:
%
\begin{enumerate}
    \item We modify our data and set
        $$
        k_i\mapsto \lfloor\nu^* k_i\rfloor,\quad n_i\mapsto \lfloor\nu^* n_i\rfloor,\quad i=1,\dots,m.
        $$
        Here $\lfloor x\rfloor, x\in\mathbb{R}$ is the largest integer $n\leq x$.
        With these modified trial numbers, we perform inference as usual, i.e. we determine confidence intervals based on bootstrap or determine
        a posterior distribution.
    \item We modify our confidence limits
        %
        $$
        c \mapsto \frac{c-\hat{\theta}}{\sqrt{\nu}} + \hat{\theta},
        $$
        %
        where $c$ is the desired confidence limit.
        For Bayesian inference, we will typically have to scale every sample $\theta^{(j)}$ according to
        %
        $$
        \theta^{(j)} \mapsto \frac{\theta^{(j)} - \hat{\theta}}{\sqrt{\nu}} + \hat{\theta}.
        $$
        %
\end{enumerate}

\end{document}
